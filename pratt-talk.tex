% Created 2016-11-15 Tue 07:12
% Intended LaTeX compiler: pdflatex
\documentclass[11pt]{article}
\usepackage[utf8]{inputenc}
\usepackage[T1]{fontenc}
\usepackage{graphicx}
\usepackage{grffile}
\usepackage{longtable}
\usepackage{wrapfig}
\usepackage{rotating}
\usepackage[normalem]{ulem}
\usepackage{amsmath}
\usepackage{textcomp}
\usepackage{amssymb}
\usepackage{capt-of}
\usepackage{hyperref}
\author{John Mignault}
\date{\today}
\title{DPLA/ESDN/Metadata}
\hypersetup{
 pdfauthor={John Mignault},
 pdftitle={DPLA/ESDN/Metadata},
 pdfkeywords={},
 pdfsubject={},
 pdfcreator={Emacs 25.1.1 (Org mode 9.0)}, 
 pdflang={English}}
\begin{document}

\maketitle

\section*{What is DPLA?}
\label{sec:org86cfb0f}
\begin{itemize}
\item The Digital Public Library of America
\begin{itemize}
\item A national digital collections project
\item Aggregates metadata from digital collections presented in their portal at \href{http://dp.la}{\url{http://dp.la}}
\end{itemize}
\end{itemize}

\section*{The Hub model}
\label{sec:orgbf61d68}
\begin{itemize}
\item Hubs supply DPLA with metadata
\begin{itemize}
\item 2 types of hub
\begin{itemize}
\item Content hubs
\begin{itemize}
\item large digital collections that provide metadata for their own content, such as NYPL or BHL
\end{itemize}
\item Service hubs 
\begin{itemize}
\item Hubs that act as a conduit for DPLA ingestion, such as ESDN
\end{itemize}
\end{itemize}
\end{itemize}
\end{itemize}

\section*{What is ESDN?}
\label{sec:org070c9b1}
\begin{itemize}
\item Empire State Digital Network
\begin{itemize}
\item NY state service hub
\item formed in 2013
\end{itemize}
\item ESDN coordinates metadata harvest, transformation, and ingest for NY state
\item ESDN provides services to state regional council (ESLN) members seeking to get their metadata into
DPLA
\item Currently funded by and based at METRO
\end{itemize}

\subsection*{Second generation hub}
\label{sec:orgfe7e1d4}
\begin{itemize}
\item Most of the first generation of DPLA hubs were digital collections websites that provided OAI feeds from their content management systems
\begin{itemize}
\item Biodiversity Heritage Library, NYPL, Digital Commonwealth (BPL)
\end{itemize}
\item ESDN part of a "second wave" of hubs
\end{itemize}

\subsection*{The "Second Wave"}
\label{sec:orgca3b179}
\begin{itemize}
\item Pure aggregators
\begin{itemize}
\item we don't host any content ourselves
\item we have no public interfaces
\end{itemize}
\item Currently we harvest partner metadata and transform it into a DPLA-approved format which they then harvest
\end{itemize}

\section*{What is our workflow?}
\label{sec:org38f6761}
\section*{Prep - Harvest - Transformation - Ingest}
\label{sec:org7be12ad}
\subsection*{Prep}
\label{sec:orga9292e6}
\begin{itemize}
\item Regional council members contact their ESDN liaison
\item The member provides a letter granting permission to use their metadata
\item The provider's contact person contacts Chris Stanton, ESDN's Metadata Specialist
\item Chris works with the provider to normalize and standardize their data
\begin{itemize}
\item Will often work on normalizations and tweaks using OpenRefine
\end{itemize}
\end{itemize}

\subsection*{Harvest}
\label{sec:org643a4cf}
\begin{itemize}
\item We then pull their data into Repox
\begin{itemize}
\item By various methods
\begin{itemize}
\item But mostly and ultimately via OAI-PMH
\item There are a few others, like CSV import into Omeka
\end{itemize}
\end{itemize}
\end{itemize}

\subsection*{Transformation}
\label{sec:org0676c67}
\begin{itemize}
\item Their data is mapped to the ESDN Metadata Application Profile (MAP)
\begin{itemize}
\item a variant of MODS
\item which is itself mapped to the DPLA MAP
\item it's an iterative process
\begin{itemize}
\item often there will be multiple rounds of writing and rewriting the transforms
\end{itemize}
\item it's a very manual process
\begin{itemize}
\item tweaking the output
\end{itemize}
\end{itemize}
\end{itemize}

\subsection*{Ingest}
\label{sec:org326f34b}
\begin{itemize}
\item Once the data has been sufficiently massaged it is ingested by DPLA
\item They perform additional transformations and enrichments
\item The result is then QA'd
\begin{itemize}
\item sometimes we need to go back and make additional adjustments
\end{itemize}
\item The records then finally go live on DPLA
\end{itemize}

\section*{Issues}
\label{sec:org28ae07e}
\subsection*{what are we trying to achieve?}
\label{sec:org399c97a}
\begin{itemize}
\item We try to strike a balance between asking the provider to normalize data and writing horrendous special cases
\item Problems are mostly dependent on how consistently the data has been entered into the CMS
\begin{itemize}
\item minor inconsistencies can be written around
\item major inconsistencies require the provider to go back and edit the data
\end{itemize}
\end{itemize}

\subsection*{Sidebar: the date field, aka "dumpster fire"}
\label{sec:orgd0c6fb8}
\begin{itemize}
\item roman numerals: MCMXVIII
\item natural language dates: June 16, 1904
\item CONTENTdm timespans: 1932;1933;1934;1935;1936 to represent 1932-1936
\item People will put anything and their dog in the date field
\end{itemize}

\section*{What is Repox?}
\label{sec:org3219041}
\subsection*{OAI aggregator software}
\label{sec:org6aae1f5}
\subsubsection*{Past}
\label{sec:org8d5d59f}
\begin{itemize}
\item originally written for the Europeana project
\item appeared to have been abandoned in 2014
\begin{itemize}
\item as such, documentation is spotty
\end{itemize}
\item used by a number of more recent DPLA hubs
\end{itemize}
\subsubsection*{Present}
\label{sec:org6d2ac5d}
\begin{itemize}
\item DPLA hub user community
\begin{itemize}
\item mailing list
\end{itemize}
\item Has a number of undocumented quirks
\begin{itemize}
\item the "records per page" quirk
\end{itemize}
\item support folklore has sprung up around it
\begin{itemize}
\item everyone uses it by necessity
\item its relative limbo makes it less than ideal
\end{itemize}
\end{itemize}
\subsubsection*{Future}
\label{sec:org5cfce3d}
\begin{itemize}
\item the long-awaited "Repox killer"
\begin{itemize}
\item DPLA has been rumored to be be working on an aggregator appliance
\item part of the Hydra in a Box project
\item rumors of its birth are greatly exaggerated
\end{itemize}
\end{itemize}

\section*{How does Repox work?}
\label{sec:orgf65f161}
\subsection*{Overview}
\label{sec:org9c65d3e}
\begin{itemize}
\item We harvest partner feeds in various formats and protocols
\item Our outgoing format is always ESDN MAP MODS
\item we define "data sets" that specify the incoming format depending on the CMS in use
\item we then write XSL stylesheets that transform the harvested data to ESDN MAP MODS
\item we attach those stylesheets to the data set
\item when DPLA ingests data they harvest the entire repository in ESDN MAP format
\end{itemize}

\subsection*{XSLT}
\label{sec:orgc9706bf}
\begin{itemize}
\item we built on the great work done at NCDHC
\item we forked their Github repository of XSLT style-sheets for use with Repox
\item NCDHC mainly works with CONTENTdm providers
\item As the number of different CMSes we saw grew, we developed a CSS-like cascading model for stylesheets
\item our repository is on Github at \href{http://github.com/esdnhub/dpla-custom-repox-xslt}{\url{http://github.com/esdnhub/dpla-custom-repox-xslt}}
\end{itemize}

\section*{Lessons}
\label{sec:org16bfd19}
\subsection*{Metadata Improvements}
\label{sec:orgde3f02e}
\begin{itemize}
\item DPLA as a national project is actually improving metadata at the state and local levels
\begin{itemize}
\item It provides an impetus for folks to address issues in their data
\item it provides justification to administrators to lend staff time to clean-up
\end{itemize}
\item it has led to the formation of the \href{http://empirestate.digital/governance/metadata-working-group/}{ESDN metadata group} (\href{http://empirestate.digital/governance/metadata-working-group/}{\url{http://empirestate.digital/governance/metadata-working-group/}}
\begin{itemize}
\item creating best practices for creating shareable metadata.
\end{itemize}
\end{itemize}

\section*{Local history, global data}
\label{sec:org4cc5d3e}
\subsection*{Data re-use}
\label{sec:org812a0c0}
\begin{itemize}
\item No public front-end
\item scope of available data constrained by MAP
\begin{itemize}
\item repox cannot link to external resources
\begin{itemize}
\item no LOD
\item limitation of XSLT
\end{itemize}
\end{itemize}
\item our partners need reporting and statistics tools
\item so, we attempted to build a basic tool
\end{itemize}

\subsection*{The "collstool"}
\label{sec:org99cdf0f}
\begin{itemize}
\item uses MODS data harvested from our Repox instance
\begin{itemize}
\item XML -> JSON -> YAML using XSLT and json2yaml
\item Jekyll reads resulting YAML file
\end{itemize}
\item ungainly, manual process that output inaccurate results
\item Running on Github Pages at \href{http://esdnhub.github.io/collstool/}{\url{http://esdnhub.github.io/collstool/}}
\item source available at \href{http://github.com/ESDNHub/collstool}{\url{http://github.com/ESDNHub/collstool}}
\end{itemize}

\subsection*{The ESDN portal}
\label{sec:org7ace97a}
\begin{itemize}
\item Plans to work on a NY state wide search portal
\item sort of a "tiny DPLA" for NY state
\item Blacklight Rails application
\begin{itemize}
\item Based on Ben Armintor's "DBLA" gem \href{https://github.com/barmintor/dbla}{\url{https://github.com/barmintor/dbla}}
\end{itemize}
\end{itemize}

\subsubsection*{How does it work?}
\label{sec:org6a2481c}
\begin{itemize}
\item makes Blacklight think it's talking to a solr repository
\begin{itemize}
\item when it's actually talking to the DPLA API
\end{itemize}
\item extended to integrate additional "external vocabularies"
\begin{itemize}
\item Sub-collection info
\item Council info
\item Possibly LCSH
\end{itemize}
\item Build additional reporting and search capabilities
\item Prototype is up at \href{https://afternoon-shelf-4328.herokuapp.com}{\url{https://afternoon-shelf-4328.herokuapp.com}}
\end{itemize}

\subsection*{DPLA Exhibitions}
\label{sec:org2e21526}
\begin{itemize}
\item Avilable at \href{http://dp.la/exhibitions}{\url{http://dp.la/exhibitions}}
\item Built on Omeka (\href{http://omeka.org}{\url{http://omeka.org}})
\item ESDN will begin providing DPLA exhibition hosting for partners
\begin{itemize}
\item Later in 2016
\end{itemize}
\end{itemize}

\section*{Questions?}
\label{sec:orgd4a06c2}
\begin{itemize}
\item Thanks!
\item John Mignault (jmignault@metro.org)
\item \href{http://metro.org}{METRO (\url{http://metro.org})}
\item \href{http://empirestate.digital}{ESDN (\url{http://empirestate.digital})}
\item \href{http://esln.org}{Empire State Library Network (esln.org)}
\item \href{http://dp.la}{DPLA (\url{http://dp.la})}
\end{itemize}
\end{document}